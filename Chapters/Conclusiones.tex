\section{Estado actual del desarrollo}
En la versión actual de la aplicación ya es posible generar y visualizar habitaciones, así como interactuar con ellas. También se puede cambiar entre los diferentes estados y hacer y deshacer acciones.

Para el desarrollo de la interfaz en la versión actual de la aplicación se ha hecho uso de la librería ImGui\footfullcite{imgui}. Se trata de una API que permite añadir diversos tipos de elementos de interfaz en la aplicación. Aunque es muy fácil de utilizar y versátil, probablemente será insuficiente de cara a la versión en producción de la aplicación: ImGui está altamente enfocado a interfaces de debugging y los requisitos estéticos de los planificadores no podrán cumplirse fácilmente. Será necesario buscar alternativas.

\section{Futuras iteraciones}
La adición de elementos interiores e interacción con estos no está terminada, aunque ya se sabe cómo se van a organizar estos elementos (véase el apartado \ref{managers}). Tampoco está hecha la interfaz de usuario y los inputs deben pulirse. Estas serán las prioridades en el futuro inmediato.

A más largo plazo, se deben implementar las diferentes aplicaciones que van a hacer uso del planificador. Se requerirá hacer aplicaciones con múltiples habitaciones y elementos estructurales como posiblemente zonas exteriores (balcones, terrazas o jardines). El planteamiento de diseñar el planificador para ser genérico hará que esto se pueda hacer con relativamente poco esfuerzo.

Es probable que, una vez conocidos los problemas que se han encontrado a lo largo del desarrollo, se de otra iteración a algunos fragmentos del código, para mejorar la eficiencia y la estructura del propio código. Algunos fragmentos del código se han desarrollado con prisa y asumiendo una cierta deuda técnica. Sin embargo, el diseño modular de la aplicación hace que sea posible realizar estas iteraciones trabajando con fragmentos aislados de código.

En su conjunto, la mantenibilidad del código es bastante buena y deberían poder incorporarse nuevos desarrolladores sin excesivo esfuerzo. Algunos fragmentos de código, en cambio, requieren un conocimiento más profundo del problema y de los elementos que intervienen en la solución. Un desarrollador descuidado podría llegar a tener problemas con dichos fragmentos por lo que existe margen de mejora en este aspecto.

\section{Uso de Emscripten como herramienta de desarrollo}
Al comenzar el desarrollo existía cierto escepticismo sobre las posibilidades de Emscripten. Llevar código de lenguajes de escritorio a plataformas web hasta hace poco ha sido algo muy poco común. Sin embargo, las primeras pruebas con el lenguaje resultaron muy sorprendentes.

Emscripten es una herramienta muy completa, y es perfectamente apto para el desarrollo de aplicaciones web y la puesta en producción de estas, incluso para aplicaciones en 3D. Es emocionante imaginar lo que será posible con la inminente aparición de nuevas herramientas como \texttt{WebAssembly}, que podrían acelerar considerablemente las aplicaciones.

El hecho de haber podido desarrollar la aplicación en C++ hará que en el futuro sea mucho más sencillo adaptarla a otras plataformas como dispositivos móviles. Las APIs de Android e iOS parecen más sencillas de adaptar para que funcionen con C++ de lo que ha sido Emscripten, por lo que aunque es de esperar que surjan dificultades, la previsión al respecto es bastante esperanzadora.

Sin embargo, es poco probable que Emscripten sea la mejor alternativa para todas las situaciones. Hay que recordar que Javascript también puede trasladarse a otras plataformas mediante wrappers. El resultado de hacer esto es mucho menos eficiente pero es más sencillo y la eficiencia no siempre es tan crítica. El punto más complicado en la adaptación del código con Emscripten ha sido la gestión de la asincronía al cargar ficheros.