\section{APIs gráficas}
\label{APIS}

Según la plataforma en que se esté ejecutando la aplicación, se hará uso de una API gráfica u otra. Por defecto Manta ha sido desarrollado para funcionar en escritorio haciendo uso de Vulkan, pero esta API no está disponible en web, por lo que tendremos que utilizar OpenGL ES 2 y WebGL en este caso.

\subsection{Vulkan}
Entre las diferentes APIs gráficas que existen, la tendencia actual es la de dar cada vez más control al desarrollador sobre lo que ocurre entre la aplicación y la gráfica, dando acceso de bajo nivel al hardware. Vulkan es la respuesta de software libre a esta tendencia, y una de las APIs que más tracción está recogiendo últimamente.

Al ofrecer control de bajo nivel, con Vulkan pueden realizarse muchas optimizaciones que en una API de alto nivel no sería posible realizar\footfullcite{vulkan_spec}. Vulkan facilita el uso de múltiples dispositivos de hardware con diversos propósitos (la GPU puede utilizarse también para realizar cálculos en paralelo, sin estar necesariamente renderizando) y el uso de multithreading.

Para ello Vulkan puede detectar y listar los dispositivos de hardware que se encuentran disponibles en la máquina, así como sus capacidades y especificaciones. Por cada dispositivo se crea una cola de comandos, que describen las acciones a realizar por el hardware.

Los comandos pueden tener distintos estados que permiten controlar el flujo de la aplicación, especialmente en aplicaciones multinúcleo.

\subsection{OpenGL ES 2 y WebGL}

\subsection{Otras APIs actuales}