\textit{Interiorvista Planner} es una herramienta que permite crear de forma rápida y fácil el diseño de una sala a medida. Está pensada para ser genérica, de modo que después el proyecto se subdivide en otras aplicaciones.

La aplicación debe permitir configurar las dimensiones y forma de una sala para después introducir elementos propios de cada proyecto en esta, para finalmente obtener un listado de los productos introducidos, el precio de comprar dicha configuración, y un código que permite acceder al proyecto desde una tienda física para realizar la compra.

\section{Clientes}
IKEA\cite{ikea_history} es una archiconocida multinacional especializada en la venta de muebles de bajo coste. Se gestó en Suecia en el año 1943 como una tienda de venta de productos varios para el día a día a un precio reducido, y cuenta hoy con 314 tiendas repartidas en 38 países, siendo el icono más reconocible en el mundo de los muebles.

Una de las claves del éxito de IKEA es su famoso catálogo, donde los potenciales clientes podían ver los muebles que se ofertan y sus posibles distribuciones. Durante los años 2000 IKEA ha puesto su catálogo a disposición de los clientes también a través de Internet, y les ha ofrecido nuevas herramientas con las que poder imaginar cómo van a quedar los productos que compren en su hogar.

ROCA\cite{roca_history} es el principal proveedor de productos para baños del mundo. Se gestó en Gavá en 1917 como una compañía de radiadores, pero su relación con el agua hizo que se interesara rápidamente en la fabricación de porcelana en 1936 y grifería en 1954.

Al igual que IKEA, ROCA ha encontrado en internet nuevas formas de llegar a sus clientes, creando catálogos online y herramientas de configuración y visualización.

Aunque estos son los dos principales clientes de Interiorvista, la naturaleza genérica del planificador hace que cualquier empresa especializada en el interiorismo sea un potencial cliente.

\section{Objetivo del producto}
Los productos de interiorismo destacan por ser altamente configurables y modulares, para adaptarse a los gustos y necesidades de cada comprador. Esto hace que sea complejo crear una aplicación que tenga en cuenta todas las peculiaridades de los productos. Con los \textit{Interiorvista Planner} los compradores deben poder probar y visualizar las diferentes configuraciones de los productos y realizar la compra (a través del código) si así lo deciden.

Por lo tanto, el objetivo es conseguir que un máximo número de clientes generen un código y lo recuperen desde una tienda (signo de que han acabado comprado el producto). Cuanto más satisfactorio sea el proceso de configuración, más probable es que dichos clientes lleguen hasta el final, es por esto que la usabilidad y los tiempos de carga son clave.

Otro objetivo indirecto es hacer que la aplicación sea lo suficientemente genérica como para poder adaptarse, con poco esfuerzo, a los diferentes sub-proyectos.

\section{Usuarios}
Hay dos posibles usuarios de la aplicación: los compradores, que pueden acceder a esta a través de la página web del cliente, desde los ordenadores disponibles en las tiendas o bien desde las aplicaciones móviles; y los empleados del cliente (también conocidos como coworkers), que se encuentran en las tiendas vendiendo productos y ayudando a los clientes.

El enfoque para cada cliente es algo distinto. En el caso de los compradores se busca algo más rápido y emocional, que le lleve lo más rápido posible a la compra del producto. Sin embargo, para los trabajadores esta aplicación es una completa y precisa herramienta de trabajo, que ha de ser capaz de poder reflejar cualquier posible configuración.

A pesar de esto la aplicación será razonablemente similar en ambos casos, a excepción de algunos ``atajos" con los que los trabajadores podrán llegar más rápido a las secciones que les interesan.
