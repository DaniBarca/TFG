\documentclass{report}
\usepackage[utf8]{inputenc}

% Style and margins:
\usepackage{geometry}
\geometry{
    a4paper,
    left=30mm,
    right=30mm,
    top=25mm,
    bottom=25mm,
    twoside
}
\setlength{\parskip}{3mm}
\usepackage{url}
\usepackage{times}
\pagestyle{plain}

\renewcommand{\contentsname}{Indice}

\usepackage{titlesec}
\titleformat{\chapter}
  {\normalfont\Huge\bfseries}{\thechapter}{1em}{}

% For figures:
\usepackage{graphicx}
\usepackage{float}
\graphicspath{ {Figures/} }
\renewcommand{\figurename}{Fig.}

% For source code:
\usepackage{listings}
\usepackage{xcolor}

\lstset{
  frame=tb,
  basicstyle=\footnotesize\ttfamily,
  language=C++,
  numbers=left,
  numbersep=5pt,
  breaklines=true,
  extendedchars=true,
  literate={ñ}{{\~n}}1
}

% Math
\usepackage{amsmath}

%Bibliography
\usepackage[style=verbose-ibid, backend=biber,sorting=none]{biblatex}
\bibliography{Bibliography}

% Data
%Desenvolupament d'aplicació 3D per escriptori i web mitjançant Emscripten
%Development of 3D application for desktop and web using Emscripten
\title{Desarrollo de aplicación 3D para escritorio y web mediante Emscripten}
\author{Daniel Barca Casafont}
\date{Junio de 2017}

%%%%%%%%%%%%%%%%%%%%%%%%%%%%%%%%%%%%%%%%%%%%%%%%%
\begin{document}

\pagenumbering{roman}
\begin{titlepage}
\maketitle
\end{titlepage}

\cleardoublepage
\tableofcontents

\cleardoublepage
\begin{abstract}
En este documento se recoge el proceso de desarrollo de una aplicación 3D multiplataforma para la empresa Interiorvista, la cual permite configurar y visualizar estancias con el lenguaje de programación C++ dados unos requerimientos inciales. Haciendo uso de Emscripten, el código ha sido adaptado a las limitaciones de las plataformas web que no encontramos en aplicaciones de escritorio tradicionales, tales como la existencia de un único hilo de ejecución o la ausencia de un sistema de ficheros. También se deberá adaptar el motor gráfico de la empresa para que funcione mediante WebGL o Vulkan en función de la plataforma donde se ejecute. Además, se discuten detalles sobre el diseño de software y el proceso de desarrollo de la aplicación en sí.

This document describes the process of developing a 3D multiplatform application for the company Interiorvista, that allows users to configure and visualize rooms with the C++ programming language given some initial requirements. Using Emscripten, the code has been adapted to the limitations of web platforms that are not found in traditional desktop applications, such as the existence of only one single execution thread or the absence of a file system. The company's graphics engine must also be adapted in order to work using WebGL or Vulkan depending on the platform where it's running. In addition, this document discusses details about the software design and the development process of the application itself.
\end{abstract}

\cleardoublepage
\pagenumbering{arabic}

\chapter{Introducción}
\section{Motivación}
Interiorvista es una empresa especializada en la generación de imágenes por computador (mucho más baratas, rápidas y de igual o mejor calidad que las que pueden obtenerse con un plató y un fotógrafo) y en el desarrollo de aplicaciones web (que requieren personal muy especializado y una gran inversión de tiempo).

Entre estas aplicaciones se encuentran los \textit{Interiorvista Planner}, un conjunto de aplicaciones que tienen el objetivo de permitir a los usuarios generar habitaciones tridimensionales y poblarlas con los productos que los clientes ofrecen en su catálogo. Esto plantea una serie de retos a varios niveles.

En su estado actual, las aplicaciones desarrolladas tienen problemas que hacen cada vez más difícil el mantenimiento y la mejora de estas. Muchas de sus características han sido desarrolladas sin llevar a cabo ningún diseño previo, o incluso sin una especificación previa de los requerimientos de la aplicación, provocando que estos surjan a lo largo del desarrollo.

\begin{figure}[H]
    \centering
    \includegraphics[width=0.8\linewidth]{bathroom_vista}
    \caption{Versión actual del planificador Bathroom Vista en vista 3D.}
    \label{fig:bathroom_vista}
\end{figure}


Las compañías de interiorismo suelen tener una serie de normas por las cuales ciertos elementos estructurales no pueden introducirse en ciertas combinaciones o en ciertas posiciones. Esto ha llevado a un código demasiado especializado en el que se introducen excepciones y condiciones arbitrarias sin mucho orden.

Se suelen requerir diversas aplicaciones muy similares para los distintos ambientes que ofrecen en su catálogo: habitaciones, comedores, baños, cocinas, etc. Aunque cada caso tiene sus particularidades, en general la mayoría de planificadores tienen suficientes características comunes como para poder tener un núcleo común, cosa que no está ocurriendo en estos momentos.

Generalmente casi siempre vamos a tener una habitación con ventanas, puertas, y una serie de elementos interiores que podemos distribuir por esta. Por ello, con un diseño efectivo debe ser posible reducir la especialización de cada una de estas aplicaciones. En el futuro, una posibilidad con la que se ha soñado en Interiorvista es la de hacer un planificador completo de una planta, con todas sus habitaciones, algo que no resultaría sencillo de conseguir con los desarrollos de que disponemos actualmente.

Entre las características comunes de los planificadores encontramos que la mayoría disponen de un modo visualización en 2 dimensiones, pensado para configurar la estructura de una habitación, y otro en 3 dimensiones, pensado para visualizar el resultado y realizar retoques sobre este. En estos momentos estos modos se han programado como dos programas distintos con una parte 2D hecha con tecnologías web, y una 3D hecha con un motor gráfico exportado a WebGL. Es una duplicidad de esfuerzos que puede evitarse utilizando una cámara ortogonal en 3D y algunas modificaciones visuales.

\begin{figure}[H]
    \centering
    \includegraphics[width=0.8\linewidth]{bathroom_vista_2d}
    \caption{Bathroom Vista, versión en dos dimensiones del baño en la figura \ref{fig:bathroom_vista}.}
    \label{fig:bathroom_vista_2d}
\end{figure}

A pesar de las similitudes siempre hay elementos que hacen único a cada Planner: algunos baños o cocinas tienen diferentes texturas combinables para las paredes, algunas habitaciones tienen un techo inclinado o algunos productos de son altamente configurables y requieren más atención. Diferentes planificadores pueden tener un enfoque distinto: desde completas herramientas que han de poder generar y visualizar todas las configuraciones posibles de una estancia hasta aplicaciones que buscan un enfoque más emocional que atraiga a los usuarios, lo cual implica sacrificar funcionalidad en favor de la estética. Incluso pueden llegar a existir dos aplicaciones para un mismo conjunto de productos, con el objetivo de cubrir ambos puntos de vista.

El hecho de estar utilizando motores gráficos propietarios nos ha provocado problemas en el pasado. Las necesidades de la empresa son muy específicas y la imposibilidad de controlar el funcionamiento del motor ha hecho que no podamos solucionar efectivamente muchos problemas, llevándonos incluso a tener que esperar a que los desarrolladores lancen actualizaciones que arreglen nuestros problemas, o a mantener versiones abandonadas porque las versiones nuevas no son compatibles con ciertos requerimientos.

La falta de control sobre el motor gráfico también ha hecho que no pudiéramos realizar ciertas mejoras de eficiencia, visualización, o reducir el peso de la aplicación (por ejemplo, las versiones actuales incluyen toda una API de audio que no es necesaria).

\section{Objetivos}

La aplicación debe ser lo bastante genérica como para aplicarse a diferentes casos, de ahí que digamos que se trata de un conjunto de aplicaciones; y al mismo tiempo también ha de ser flexible como para dar cabida a todas estas características.

Debe contar con un visualizador en 2 y 3 dimensiones. Según el modo de visualización la interacción y las opciones son diferentes, pero el estado y la lógica de la aplicación debe mantenerse el máximo posible.

A largo plazo, la aplicación debe poder ejecutarse sobre distintas plataformas como web, escritorio, móviles o tabletas. Esto tiene muchas implicaciones a nivel de software e interacción: distintas plataformas cuentan con distintos drivers y APIs de ejecución y cada una tiene un funcionamiento y un modo de uso muy distinto. Dado que la aplicación va a estar completamente programada en C++, trasladar el código a otras plataformas (especialmente web) puede suponer un reto.

El desarrollo se realizará sobre un motor gráfico propio de la empresa, el cual está pensado para funcionar con la API gráfica Vulkan en escritorio, y debe ser adaptado para poder utilizarse con otras APIs en otras plataformas como Web o plataformas móviles. El hecho de disponer de un motor gráfico propio nos da un gran control sobre lo que ocurra dentro de este, en contraste con otras alternativas propietarias que no podemos controlar. En el pasado se han tenido muchos problemas haciendo funcionar las aplicaciones en distintas plataformas.

A nivel de diseño de software, esta es una oportunidad para repensar y reorganizar los problemas que ya conocemos. Aplicar correctamente diversas técnicas de diseño de software hará que no sólo sea más sencillo desarrollar la aplicación sino que sea más fácil de mantener y ampliar en el futuro. Algunas características son muy difíciles de implementar si no se ha seguido un cierto diseño desde el principio.

\section{Estado del arte}
Aunque existen diversos planificadores de estancias en el mercado, a día de hoy la mayoría tienen serias deficiencias y prácticamente ninguno está asociado a marcas importantes del modo en que lo está Interiorvista. Sin embargo, eso no impide que aprendamos de las alternativas existentes.

Entre los fallos más comunes se encuentran:
\begin{itemize}
    \item La necesidad de descargar aplicaciones de escritorio, o un gran número de assets que no necesariamente van a utilizarse.
    \item El uso de tecnologías obsoletas, especialmente Adobe Flash (muy popular durante la última década pero en desuso hoy en día), o motores web que requieren la instalación de plugins o extensiones.
    \item Sólo modo en 2 dimensiones o 3 dimensiones, sin la posibilidad de cambiar.
    \item Mala calidad gráfica.
    \item Interacción y/o diseño pobre.
\end{itemize}

Por supuesto, tenemos como precedente los anteriores planificadores hechos en Interiorvista, que aunque están bien situados en el mercado sufren de algunos de los fallos ya mencionados. La alternativa más sólida para lo que queremos realizar es Planner5D\cite{planner5d}, que cumple buena parte de los requerimientos que queremos cumplir; sin embargo a día de hoy también tiene defectos en estabilidad e interacción, como que los elementos interiores no se adhieren a las paredes (a excepción de elementos estructurales de las paredes, como puertas y ventanas), o que en 2D pueden estropearse las paredes de forma relativamente fácil.

%Los planificadores de 2020\cite{2020} se caracterizan entre otras cosas por ofrecer imágenes estáticas de mayor calidad. El planificador espera a que el usuario deje de interactuar con la escena para generar la imagen y sobreponerla al canvas 3D. Aunque funciona, probablemente no se trata de la mejor aproximación a este problema, dado que el resultado no deja de ser un tanto pobre para la cantidad de trabajo que han realizado: generar un render 3D de buena calidad en la nube tiene un gran coste computacional (y por extensión, económico, pues los servidores gráficos son especialmente costosos), por lo que generar un render nuevo cada pocos segundos es inviable. Para solventarlo han reducido la calidad igualmente haciendo que, si bien se ve mejor que el render local, sigue dejando mucho que desear. Además, a nivel de interacción puede resultar engorroso notar el cambio de calidad constantemente.

%Generar imágenes de alta calidad en tiempo real no deja de ser una opción interesante, pero vale la pena considerar otras como mantener el renderizado local hasta que el usuario haya terminado de configurar su habitación, para finalmente generar una o varias imágenes de alta calidad.

\cleardoublepage
\chapter{Descripción del proyecto}
\textit{Interiorvista Planner} es una herramienta que permite crear de forma rápida y fácil el diseño de una sala a medida. Está pensada para ser genérica, de modo que después el proyecto se subdivide en las aplicaciones Bath, Bed y Metod, aunque se podrán crear otros proyectos que tengan características similares.

La aplicación debe permitir configurar las dimensiones y forma de una sala para después introducir elementos propios de cada proyecto en esta, para finalmente obtener un listado de los productos introducidos, el precio de comprar dicha configuración, y un código que permite acceder al proyecto desde una tienda física para realizar la compra.

\section{Empresa}
Interiorvista es una empresa con sede en Igualada, especializada en la generación de imágenes por computador (mucho más baratas, rápidas y de igual o mejor calidad que las que pueden obtenerse con un plató y un fotógrafo) y en el desarrollo de aplicaciones web (que requieren personal muy especializado y una gran inversión de tiempo). El mejor ejemplo de aplicación web lo encontramos en los planificadores: herramientas pensadas para que los potenciales compradores puedan configurar y visualizar los productos del cliente.

\section{Clientes}
IKEA es una archiconocida multinacional especializada en la venta de muebles de bajo coste. Se gestó en Suecia en el año 1943 como una tienda de venta de productos varios para el día a día a un precio reducido, y cuenta hoy con 340 tiendas repartidas en 28 países, siendo el icono más reconocible en el mundo de los muebles.

Una de las claves del éxito de IKEA es su famoso catálogo, donde los potenciales clientes podían ver los muebles que se ofertan y sus posibles distribuciones. Durante los años 2000 IKEA ha puesto su catálogo a disposición de los clientes también a través de Internet, y les ha ofrecido nuevas herramientas con las que poder imaginar cómo van a quedar los productos que compren en su hogar.

ROCA es el principal proveedor de productos para baños del mundo. Se gestó en Gavá en 1917 como una compañía de radiadores, pero su relación con el agua hizo que se interesara rápidamente en la fabricación de porcelana en 1936 y grifería en 1954.

Al igual que IKEA, ROCA ha encontrado en internet nuevas formas de llegar a sus clientes, creando catálogos online y herramientas de configuración y visualización.

Aunque estos son los dos principales clientes de Interiorvista, la naturaleza genérica del planificador hace que cualquier empresa especializada en el interiorismo sea un potencial cliente.

\section{Objectivo del producto}
Los productos de interiorismo destacan por ser altamente configurables y modulares, para adaptarse a los gustos y necesidades de cada comprador. Esto hace que sea complejo crear una aplicación que tenga en cuenta todas las peculiaridades de los productos. Con los Interiorvista Planner los compradores deben poder probar y visualizar las diferentes configuraciones de los productos y realizar la compra (a través del código) si así lo deciden.

Por lo tanto, el objetivo es conseguir que un máximo número de clientes generen un código y lo recuperen desde una tienda (signo de que han acabado comprado el

producto). Cuanto más satisfactorio sea el proceso de configuración, más probable es que dichos clientes lleguen hasta el final, es por esto que la usabilidad y los tiempos de carga son clave.

Otro objetivo indirecto es hacer que la aplicación sea lo suficientemente genérica como para poder adaptarse, con poco esfuerzo, a los diferentes sub-proyectos.

\section{Usuarios}
Hay dos posibles usuarios de la aplicación: los compradores, que pueden acceder a esta a través de la página web del cliente, desde los ordenadores disponibles en las tiendas o bien desde las aplicaciones móviles; y los empleados del cliente (también conocidos como coworkers), que se encuentran en las tiendas vendiendo productos y ayudando a los clientes.

El enfoque para cada cliente es algo distinto. En el caso de los compradores se busca algo más rápido y emocional, que le lleve lo más rápido posible a la compra del producto. Sin embargo, para los trabajadores esta aplicación es una completa y precisa herramienta de trabajo, que ha de ser capaz de poder reflejar cualquier posible configuración.

A pesar de esto la aplicación será razonablemente similar en ambos casos, a excepción de algunos ``atajos" con los que los trabajadores podrán llegar más rápido a las secciones que les interesan.


\cleardoublepage
\chapter{Motor gráfico}
Como se ha mencionado anteriormente, el motor gráfico con el que trabajaremos para crear esta aplicación es propio de la empresa: Manta. Manta está programado en C++ y pretende ser un motor multipropósito, aunque el hecho de estar programado en la misma empresa nos permite prestar especial atención a los usos específicos que le demos dentro de esta. En este apartado discutiremos algunas de las características más relevantes del motor para con la aplicación. No se pretende pues crear una documentación completa de este sino tan sólo una visión general.

\section{Diseño y estructura del motor}

Antes de empezar a trabajar debemos conocer cómo vamos a comunicarnos con el motor gráfico y tener al menos una buena idea de la manera en que tratará la información que le proporcionemos.

\subsection{Ratio de fotogramas por segundo y bucle de ejecución}
\label{fps_bucle_ejecucion}
Aunque puede funcionar para la generación de imágenes estáticas, Manta es sobretodo un motor gráfico en tiempo real. Esto significa que su código se ejecuta múltiples veces para generar imágenes distintas y mostrarlas en pantalla, con lo cual se logra una ilusión de movimiento. De manera óptima suele considerarse como estándar los 60 fotogramas por segundo\cite{vsync_nvidia} (FPS o frames en adelante), aunque 24-30 FPS suele considerarse el mínimo aceptable. Debe tenerse en cuenta que aunque estos son los valores habituales, la tolerancia varía según el tipo de aplicación: en nuestro caso aunque buscamos una respuesta fluida, puede llegar a ser aceptable una bajada de los FPS sin afectar gravemente la experiencia de usuario.

Como implicación tenemos que para una buena experiencia en tiempo real todos los aspectos de la aplicación, incluyendo los cálculos de la propia aplicación como del propio motor y el renderizado a través de la GPU, deben calcularse como mínimo en unos $0.04$ segundos y óptimamente en $0.016$ segundos (los mencionados 24 y 60 FPS). En contraste, un render de alta calidad con ray tracing (una técnica que simula la física detrás de la interacción entre la luz y las superficies de un espacio) puede tardar varios minutos u horas\cite{nvidia_raytr}. Es de esperar por tanto que el motor sacrifique gran parte de ese realismo para reducir el tiempo de ejecución.

Para hacer esto de manera indefinida se encapsula todo el código del programa en un bucle de infinito que sólo puede detenerse manualmente y que contiene, en orden:

\begin{itemize}
    \item El cálculo del tiempo de ejecución del frame: las variaciones de FPS que puedan producirse a lo largo de la ejecución pueden provocar un efecto de aceleración y deceleración que empeora notablemente la experiencia. Además no se puede predecir cual será el ratio de FPS al que se ejecutará la aplicación cuando no conocemos en qué máquina se ejecutará y cual es su potencia. Para evitar este tipo de indeterminación calculamos el tiempo que transcurre entre un frame y el siguiente, el cual puede utilizarse en los cálculos a la hora de actualizar para compensar.
    \item La actualización de la aplicación: se llama una función desde la cual debemos actualizar la escena en función de los cálculos que realicemos. Cuando la aplicación alcance cierta complejidad, este puede llegar a ser el punto del bucle que absorba una mayor carga de trabajo, por lo que la eficiencia debe ser tenida en cuenta al actualizar para no ralentizar demasiado el programa.
    \item La actualización de la escena: Como se mencionará en el apartado \ref{scene_hierarchy}, el motor dispone de una colección de elementos que le hemos ordenado mostrar en pantalla. En este apartado se actualiza la escena para que refleje los cambios hechos en el proceso de actualización de la aplicación. Entre otras cosas, en este punto se calculan las transformaciones de las entidades (\ref{entity_component}) y se prepara toda la información que pueda necesitarse para renderizar.
    \item Renderizado: En este punto se envía toda la información necesaria a la GPU (incluyendo la cámara, las luces y los elementos de la escena) y se le ordena renderizar la escena. Aquí predomina el funcionamiento de la API gráfica, por lo que según la plataforma en que nos encontremos, este paso será realizado por una API diferente (véase el apartado \ref{APIS}).
\end{itemize}

Previamente al bucle se ejecutan las funciones de inicialización tanto del motor como de la aplicación, mientras que al detener la ejecución se libera la memoria reservada (normalmente durante esta primera inicialización).

\begin{figure}[H]
    \centering
    \includegraphics[scale=0.70]{Bucle}
    \caption{Esquema del bucle de ejecución.}
    \label{fig:bucle}
\end{figure}

\subsection{Patrón de entidad componente}
\label{entity_component}
En una aplicación 3D, podemos entender como entidad cualquier elemento que se encuentre en la escena. Por lo general todas tienen un elemento en común: tienen una transformación que indica su traslación, rotación y escalado. Sin embargo, cada entidad puede tener propósitos distintos: algunas son luces, otras cámaras, o objetos tridimensionales con malla y material, etc.

Dentro de cada categoría pueden haber varios tipos, o tal vez necesitamos que una entidad cumpla unas propiedades determinadas como emitir audio o verse afectada por un motor de físicas; y también podemos querer combinaciones de estas como que un elemento sea una luz y al mismo tiempo se balancee mediante un motor de físicas.

No queremos que nuestro código mezcle cosas tan dispares, tener las físicas y el audio en el mismo sitio resultaría en un código imposible de mantener a largo plazo; por lo que tenemos que buscar un modo de modularizar estas características. Para ello existe el patrón entidad-componente\cite{game_programming_patterns}, que nos permite crear componentes que aglomeran ciertas propiedades y comportamientos. Después podemos añadir cuantos componentes queramos a una entidad, permitiendo hacer que esta cumpla dichas propiedades sin aglomerar todo el código de estas.

Para implementar este patrón tenemos dos clases principales "Entity" y "Component", las cuales podremos extender como deseemos para crear distintos tipos de cada una. Una entidad por defecto contiene una lista de componentes, que en el fondo son instancias de los diferentes tipos de componentes que creemos. Component es una clase abstracta y sus clases derivadas deben implementar uno o varios métodos que puedan ser llamados desde la entidad.

De este modo, cada vez que la entidad necesite delegar una funcionalidad a sus componentes, iterará sobre ellos y llamará dichos métodos, que tendrán un comportamiento distinto según se requiera. En nuestro caso, los métodos que implementa Component son los siguientes:

\begin{itemize}
    \item AddedToEntity: Llamado en el momento en que se incorpora el componente a la entidad.
    \item RemovedFromEntity: Llamado en el momento en que este se borra de la entidad (lo cual normalmente significa que estamos eliminando la entidad, pero no necesariamente).
    \item AddedComponent: Se llama a todos los componentes que tenga la entidad cada vez que se añade uno nuevo, e incluye una referencia por parámetro del componente añadido.
    \item RemovedComponent: Al igual que la anterior, cuando se elimina un componente también se hace saber al resto de componentes aún existentes.
\end{itemize}

Aunque no lo hemos utilizado, es bastante típico en el patrón Component tener un método "update" que se llama en cada actualización de la aplicación (\ref{fps_bucle_ejecucion}) y hace que las propias entidades se encarguen de su propio comportamiento en tiempo real, a través de los componentes. Nosotros no hemos implementado esta capacidad porque hemos decidido que sean funciones externas quienes controlen dicho comportamiento, y no las propias entidades.

Para acceder externamente a los compoentes, en la clase Entity hacemos uso de "templates" de C++, sobre los cuales no vamos a profundizar. Los templates permiten implementar un método independientemente del tipo de las variables con las que trabaja, de modo que este tipo se especifica en el momento de llamar al método. Para pedirle a una entidad que nos dé acceso a un componente específico, le podemos especificar el tipo del componente que queremos y desde el método buscar qué componentes son de dicho tipo.

\subsection{Escena, jerarquía y pasos previos al renderizado}
\label{scene_hierarchy}
Como se ha mencionado en el apartado \ref{entity_component}, disponemos de una serie de entidades distribuidas por el espacio con una traslación, rotación y escalado. Este espacio se conoce como la escena.

Al crear una entidad, esta automáticamente se registra a sí misma en la jerarquía de la escena. La jerarquía contiene por tanto un listado con punteros a cada una de las entidades que le permite acceder a sus componentes. El motor trabaja directamente con este listado de objetos y hará directamente todo lo que se requiera con ellos dependiendo de sus componentes. Por lo tanto, el usuario no debe pedir en ningún momento que se rendericen los objetos: basta con crear una entidad que tenga un componente de malla (véase el apartado \ref{mesh_light_cam}) para que el motor entienda que deberá renderizarlo.

Las entidades están organizadas en la jerarquía de tal modo que una entidad puede ser padre de otras entidades, formando una estructura de árbol. Las transformaciones de las entidades se acumulan desde la raíz de la jerarquía hacia abajo, es decir: si una entidad está desplazada o rotada, sus entidades "hijas" tendrán una traslación y rotación respecto a la primera. Esto permite crear diferentes estructuras dentro de la escena y trabajar con estas sin tener que controlar la posición de todos los elementos que la componen. Por ejemplo: si tenemos una entidad "coche" y otra entidad "asientos" asignar el primero como padre de los segundos hará que se mantengan en su sitio, pegados a la estructura del coche, dado que su transformación es relativa a este.

Una consecuencia de este sistema es que las transformaciones que tenemos en cada entidad no son realmente la transformación respecto al centro de la escena, que es la que necesitamos en el momento de renderizar. Por lo tanto debemos realizar un cálculo previo al renderizado que consiste en multiplicar las transformaciones en cada una de las ramas de la escena y asignarlas a cada uno de los elementos.

\subsection{Mallas, luces y cámara}
\label{mesh_light_cam}
Se trata de los ejemplos más importantes de componentes (véase el apartado \ref{entity_component}) que se utilizan en el motor.

\subsubsection{Mallas}
Entendemos por malla (o mesh) un listado de vértices conectados por otro listado de índices. Los vértices contienen información de la posición, normal, tangente, bitangente y coordenadas UV (véase \ref{materials}) mientras que los índices simplemente ordenan los vértices de 3 en 3 creando triángulos. Los componentes "MeshComponent" y "MeshDynamicComponent" recogen esta información para que el motor renderice el resultado posteriormente. Estos componentes también incluyen los materiales de las mallas.

La diferencia principal entre ellos es que "MeshComponent" lee los datos de un fichero al crearse y es inmutable, mientras que "MeshDynamicComponent" se crea asignando una cantidad máxima de vértices e índices y estos se asignan programáticamente, pudiendo modificarse en cualquier momento (será muy importante para la generación dinámica de paredes y ventanas en la sección \ref{walls_holes}). Otra diferencia importante es que "MeshComponent" puede contener diversas sub-mallas que comparten transformación pero pueden tener distintos materiales, mientras que "MeshDynamicComponent" solo puede contener una.

\subsubsection{Luces}
Las luces son el elemento más complejo de un motor gráfico. La calidad de la luz es el elemento que más influye en el realismo de la imagen generada, y es uno de los puntos que más coste computacional requiere. En las técnicas de renderizado de alta calidad, se trata de simular la física de la luz para conseguir un gran realismo, pero esto es impracticable en un motor en tiempo real como se ha dicho en el apartado \ref{fps_bucle_ejecucion}.

En el momento de renderizar se utilizan luces para saber con qué intensidad debe renderizarse cada elemento de la escena, o los diferentes puntos de su superficie. Conociendo en qué dirección incide la luz y la normal de la superficie en el punto que queremos pintar, mediante el producto escalar de estos vectores podemos saber si la luz incide sobre este punto y con qué intensidad. Además las luces pueden tener color (normalmente será luz blanca pero no tiene por qué ser así) que se refleja mezclando el color de la luz con el de la superficie.

En la realidad las luces pueden tener formas muy variadas, pero en el motor se reducen a: puntos de luz, luces direccionales, focos, luces de área, luces esféricas y luces cilíndricas. El tipo de luz hará variar el modo en que incide sobre la escena. Las propiedades de esta se asignan en el momento de crear el componente y pueden cambiarse en cualquier momento.

\subsubsection{Cámara}
La cámara indica el punto de vista desde el que se realizará el renderizado de la escena. 

\subsection{Materiales}
\label{materials}
El material de un objeto 3D describe el aspecto que ha de tener la superficie de este en el momento de renderizarlo.

\subsection{Gestión de memoria}
\section{APIs gráficas}
\label{APIS}

Según la plataforma en que se esté ejecutando la aplicación, se hará uso de una API gráfica u otra. Por defecto Manta ha sido desarrollado para funcionar en escritorio haciendo uso de Vulkan, pero esta API no está disponible en web, por lo que tendremos que utilizar OpenGL ES 2 y WebGL en este caso.

\subsection{Vulkan}
Entre las diferentes APIs gráficas que existen, la tendencia actual es la de dar cada vez más control al desarrollador sobre lo que ocurre entre la aplicación y la gráfica, dando acceso de bajo nivel al hardware. Vulkan es la respuesta de software libre a esta tendencia, y una de las APIs que más tracción está recogiendo últimamente.

Al ofrecer control de bajo nivel, con Vulkan pueden realizarse muchas optimizaciones que en una API de alto nivel no sería posible realizar\footfullcite{vulkan_spec}. Vulkan facilita el uso de múltiples dispositivos de hardware con diversos propósitos (la GPU puede utilizarse también para realizar cálculos en paralelo, sin estar necesariamente renderizando) y el uso de multithreading.

Para ello Vulkan puede detectar y listar los dispositivos de hardware que se encuentran disponibles en la máquina, así como sus capacidades y especificaciones. Por cada dispositivo se crea una cola de comandos, que describen las acciones a realizar por el hardware.

Los comandos pueden tener distintos estados que permiten controlar el flujo de la aplicación, especialmente en aplicaciones multinúcleo.

\subsection{OpenGL ES 2 y WebGL}

\subsection{Otras APIs actuales}
\section{Exportación a otras plataformas}

\subsection{Compatibilidad de APIs gráficas}

\subsection{Compilación a escritorio}

\subsection{Compilación a plataformas móviles}
\section{Shaders}
\label{shaders}

Los shaders son programas diseñados para ejecutarse en la tarjeta gráfica\footfullcite{ogl_superbible}. A pesar de estar muy limitados, tienen la ventaja de aprovechar muy bien la capacidad de procesado paralelo de la GPU. Hoy en día los shaders son el núcleo alrededor del cual gira el renderizado de una aplicación 3D en tiempo real, y ocupan la mayor parte del tiempo de computación.

Aunque no se profundizará en lenguajes de shading, es importante saber que OpenGL ES utiliza el lenguaje de shading GLSL (OpenGL Shading Language)\footfullcite{khronos_glsl}. GLSL es un lenguaje basado en C que se compila en tiempo de ejecución mediante la API de OpenGL, a partir del propio código en texto plano. Vulkan en cambio utiliza el lenguaje intermedio SPIR-V (Standard Portable Intermediate Representation)\footfullcite{khronos_spir}. SPIR-V es el resultado de compilar GLSL mediante glslang\footfullcite{glslang} \footfullcite{spir_introduction}, y se provee en tiempo de ejecución a la API de Vulkan. Al haber sido procesado previamente SPIR-V carga sensiblemente más rápido en tiempo de ejecución. También es posible utilizar SPIR-V en OpenGL mediante una extensión (refiriéndonos a la API de escritorio, no su versión para sistemas embebidos)\footfullcite{opengl_spirext}.

El hecho de que ambos partan del mismo lenguaje de shading será significativo a la hora de adaptar el código a WebGL.

En cada iteración, se ejecuta por cada elemento en escena la pipeline de la API gráfica. La pipeline recibe como parámetro una serie de primitivas (puntos, líneas o polígonos) y ejecuta los shaders pasándoles esa información. En OpenGL, la pipeline puede simplificarse como se ve en la figura \ref{fig:opengl_pipeline}.

\begin{figure}[H]
    \centering
    \includegraphics[width=0.80\linewidth]{ogl_pipeline}
    \caption{Esquema de la pipeline gráfica de OpenGL, adaptada directamente de OpenGL Superbible\footfullcite{ogl_superbible}.}
    \label{fig:opengl_pipeline}
\end{figure}

De esta lista, los shaders marcados con una lista discontinua no son programables, sino que tienen un comportamiento predefinido. A parte de esos, los tipos que se han utilizado para el desarrollo de la aplicación son el Vertex Shader y el Fragment Shader, dado que son compatibles con WebGL.

\subsection{Vertex Shader}
El Vertex Shader\footfullcite{ogl_superbible} se ejecuta independientemente por cada uno de los vértices de la malla a renderizar. Desde este shader se pueden modificar los atributos de cada uno de los vértices, recibidos desde la etapa de ``Vertex Fetch", como su posición, color, coordenadas de textura, o cualquier otra propiedad que arbitrariamente le hayamos atribuido. También pueden añadirse nuevos atributos si se desea. Una vez modificados, estos atributos se pasan como outputs del shader para ser utilizados en otros estadios de la pipeline, como el Fragment Shader.

A la hora de programar el Vertex Shader es muy importante tener en cuenta que sus datos de salida van a ser interpolados en el Fragment Shader, y también que desde un Vertex Shader no se puede acceder a los atributos de un vértice distinto al que se está procesando.

\subsection{Fragment Shader}
El Fragment Shader\footfullcite{ogl_superbible} se ejecuta por cada píxel en pantalla que ocupe cada polígono. Sus atributos de entrada son los que se han definido como salida en el Vertex Shader pero están interpolados, es decir, entre los diferentes vértices de un polígono, los píxeles de su interior tienen los atributos intermedios de cada uno de los vértices, según su proximidad a estos. También es posible enviar datos que sean únicos para todos los píxeles, como la posición de la cámara o de las luces.

Desde el Fragment Shader podemos definir el color de un píxel específico, normalmente con la ayuda de los atributos interpolados. Se utiliza especialmente para definir la cantidad y color de luz que recibe cada fragmento del polígono, en función de factores como el ángulo entre la normal de la superficie a pintar y la dirección de la luz o el ángulo respecto a la cámara.

\cleardoublepage
\chapter{Compilación a web con Emscripten}
\label{emscripten}
\section{Exportación a Web}
Uno de los puntos fuertes de nuestro planner es el poder ejecutarlo en web. El hecho de estar programado en C++, sin embargo, dificulta esta tarea. Teniendo cuenta una serie de consideraciones, es posible adaptar código C++ a web a través de Emscripten.

\subsection{Sobre Emscripten}
Emscripten es capaz de compilar código LLVM (Low Level Virtual Machine) a Javascript. LLVM es un lenguaje de código máquina que se puede generar compilando desde otros lenguajes, entre ellos C++. Dado que Emscripten compila desde LLVM (y no C++ directamente) realmente esto significa que podemos ejecutar código escrito en diversos lenguajes a través de Emscripten.

Emscripten\footfullcite{kripken_cppcon} nació como respuesta a la necesidad de escribir programas para web en lenguajes tradicionales de escritorio, con el objetivo de ganar velocidad, trabajar con un lenguaje familiar, y aprovechar las librerías existentes en dichos lenguajes. Hasta su aparición la única alternativa para poder hacer esto era el uso de plugins embebidos en el navegador, lo cual fracasó debido a la falta de estandarización y a la evolución de los propios navegadores, que cada vez más rechazan los plugins desde la popularización de las plataformas móviles.

Dada la extensión de funcionalidades de Javascript, para realizar el compilado se ha utilizado un subset de este, ``asm.js". Asm.js es un lenguaje intermedio que podemos ejecutar en navegadores; tiene una serie de restricciones con respecto a Javascript, como por ejemplo el hecho de ser un lenguaje tipado (aunque Javascript no lo es, en asm.js se comprueba el tipo de cada variable antes de trabajar con ella) o el requerimiento de trabajar sobre una única pila de memoria (un simple array de Javascript dentro del cual deben estar todos los datos que utilice el programa).

Si se aplica esta filosofía al código que programamos en C++, se puede entender fácilmente que dicha pila de memoria equivale a la memoria tal y como la ve un programa tradicional en C++. Es posible reservar memoria dentro de esta pila, y tener punteros a posiciones de la pila (que en realidad no son más que simples índices). Tiene un funcionamiento similar al Memory Pool descrito en el apartado \ref{engine_memory}.

Aunque en cuanto a eficiencia el código en asm.js en web está en torno al 50-67\% de su equivalente en escritorio, los resultados pueden ser sorprendentes pues asm.js es capaz de superar en velocidad a su equivalente escrito en Javascript tradicional. Al trabajar sobre datos estáticos y memoria pre-inicializada, el código resultante cuenta con una serie de optimizaciones que un programador no suele realizar manualmente.

Una de las actuales tendencias del web es la ejecución de código máquina, para lo cual está en fase de desarrollo la tecnología ``WebAssembly". Sin embargo, aún a día de hoy no está del todo claro cuando estará lista lo suficientemente avanzada\footfullcite{emscripten_ready}, por lo que el papel de asm.js es importante como paso intermedio, para que los desarrolladores puedan ir adoptando la tecnología antes de ser funcional del todo. Emscripten también tiene un rol muy importante en este proceso, porque es el encargado tanto de compilar a asm.js hoy como lo será de compilar a WebAssembly en el futuro\footfullcite{webasm_roadmap}.

Emscripten ya se está utilizando hoy en día en diversos proyectos de cierta categoría, especialmente en aplicaciones gráficas.

Es importante también tener en cuenta algunos de los defectos de utilizar Emscripten respecto a programar directamente en Javascript:

\begin{itemize}
    \item Es necesario saber desde el primer momento cuanta memoria se va a necesitar. Aunque superar dicha memoria no hará que el programa falle (Emscripten es capaz de reasignar memoria cuando se sobrepasa el límite), permitir que esto ocurra no es en absoluto recomendable, puesto que ralentizaría el programa.
    
    \item Se requiere un proceso de adaptación considerable: muchas de las cosas que el programador da por sentadas en C++ no se cumplen en Javascript. Como se verá a lo largo del capítulo \ref{emscripten}, no se puede inicializar el programa, gestionar los ficheros o ejecutar el bucle del programa del mismo modo en que se haría en C++.
    
    \item Se pierde poder de decisión sobre algunas de las tecnologías que se pueden querer utilizar porque simplemente no están disponibles, como versiones modernas de OpenGL o algunas operaciones SIMD. Esto cambiará en el futuro según avance la tecnología, pero a día de hoy es un limitante.
    
    \item Si se desea que el mismo código se pueda ejecutar en diversas plataformas, habrá otro proceso de adaptación considerable para hacer que el programa tenga en cuenta las consideraciones mencionadas según la plataforma. En realidad esto es más bien un defecto de la programación multi-plataforma que de Emscripten, pero se debe tener en cuenta que existirá un esfuerzo extra de desarrollo y un aumento de la complejidad del código.
    
    \item El debugging resulta sensiblemente más incómodo que con Javascript o C++ en escritorio\footfullcite{emscripten_debugging}. Normalmente para seguir los errores se hace uso de impresiones a través de la consola y el stack-trace que devuelve Emscripten cuando se produce un error (el cual no siempre es suficientemente descriptivo).
\end{itemize}

\subsection{Adaptación de la API gráfica}
Como se ha podido ver en el apartado \ref{APIS}, el motor está pensado para funcionar utilizando la API gráfica Vulkan. El primer paso para hacer funcionar el programa con Emscripten ha sido utilizar una API gráfica que sea compatible con las tecnologías web.

WebGL está basado en OpenGL ES 2, y Emscripten puede reconocer las llamadas a esta segunda API y transformarlas en llamadas a WebGL. De modo que para hacer funcionar el motor ``bastará" con buscar los puntos donde se interactúe con la API de Vulkan y cambiarlos para que funcionen con OpenGL ES 2.

\textcolor{red}{(((AQUÍ FALTAN COSAS, PETER)))}

\subsection{Comunicación C++/Javascript}
\label{emscripten_comm}
Como es de esperar, es posible comunicar el código de Javascript con el código compilado desde C++\footfullcite{kripken_interacting}. En el desarrollo de la aplicación en C++ pueden definirse funciones pensadas para ser llamadas desde Javascript, y en el momento de compilar deben especificarse cuáles son esas funciones. Posteriormente, el Javascript resultante al compilar pone a disposición del desarrollador las funciones ``cwrap" i ``ccall".

La función ``cwrap" provee una función intermedia que se puede llamar como si de una función normal de Javascript se tratara. Esta función se encargará de llamar a su vez a la función de C++ correspondiente. En cambio, ``ccall" permite llamar directamente a la función de C++, pero como se explica a continuación es más verboso, por lo que si la función va a ser llamada varias veces puede ser más recomendable utilizar la primera alternativa.

Ambas opciones requieren que se especifique la firma de las funciones que van a ser llamadas. La firma de una función define el tipo de los parámetros que recibe y el tipo del valor de devuelve. De ahí que se diga que ``ccall" es más verboso: cada vez que se llame es necesario especificar la firma mientras que con ``cwrap" solo se debe hacer una vez.

Otro detalle importante a tener en cuenta es que no cualquier función puede ser llamada desde Javascript. Emscripten sólo permite que llamemos a funciones propias del lenguaje C, dado que C++ altera el nombre de las funciones al compilarlas (C++ permite a dos funciones llamarse igual si tienen firmas distintas, pero lo resuelve cambiando los nombres en el compilado resultante). Para garantizar que las funciones sean propias de C cuando programamos en C++ se utiliza el prefijo \texttt{extern "C"} antes de la definición.

Como se verá en el apartado \ref{init_emscripten} la comunicación entre C++ y Javascript es importante para controlar el flujo de la aplicación, pero también lo será para que desde el lado web se puedan crear interacciones que tengan efecto en el planificador. Con HTML y Javascript se pueden crear toda clase de eventos que pueden transmitirse a C++, como elementos de interfaz. Además, el planificador no controla los datos relacionados con el dominio de la aplicación, como por ejemplo qué muebles están disponibles en qué países; esta información se transmitirá a la aplicación a través de la web.

\subsection{Inicialización y bucle de ejecución}
\label{init_emscripten}
Uno de los inconvenientes de Emscripten es que Javascript ejecuta todos los procesos relacionados con la página en un único hilo de ejecución, incluyendo el renderizado, la gestión de eventos y la actualización de la propia página.

Javascript tiene una naturaleza orientada a eventos\footfullcite{javascript_queue}: dispone de una cola de eventos que se van añadiendo y ejecutando en un orden difícil de predecir. Entre esos eventos estará la propia ejecución del código Javascript y en consecuencia el código generado con Emscripten, pero también el resto de eventos mencionados que gestionan el comportamiento de la página. Si uno de los eventos tarda demasiado en terminar, bloqueará la cola de eventos y provocará que toda la página se bloquee.

Como el desarrollador no sabe en que orden se ejecutan los eventos, se dice que el lenguaje es asíncrono, y eso implica que se debe programar con cuidado para controlar que todo ocurra en el orden que se espera. La naturaleza asíncrona de Javascript puede confundir al programador haciéndole pensar que puede utilizar más de un hilo de ejecución (dado que en otros lenguajes la programación multi-hilo y la asincronía están estrechamente relacionadas); es importante tener presente que no es posible realizar dos tareas al mismo tiempo en este lenguaje. Sí se espera, en cambio, que WebAssembly pueda en el futuro ejecutar programas multi-hilo.

Como se ha dicho en el apartado \ref{emscripten_comm}, el flujo de la aplicación debe controlarse desde Javascript. El bucle de ejecución (\ref{fps_bucle_ejecucion}) es un bucle infinito, por lo que si lo ejecutamos estando dentro de C++ bloquearemos por completo la aplicación impidiendo que la cola avance. Por lo tanto el primer paso para hacer funcionar la aplicación pasa por hacer que las funciones ``Update" y ``Draw" del motor se llamen desde Javascript.

Para conseguirlo se debe utilizar la función ``setInterval"\footfullcite{setInterval}, que dado un intervalo de tiempo añade un evento para llamar a una función repetidamente. Especificando un intervalo de $1/6$ segundos, podemos hacer que la función de ``Update" se llame 60 veces por segundo. Es importante tener en cuenta que setInterval no garantiza que la función se llame realmente en el intervalo dado, sino que ese el mínimo de tiempo que se tardará en llamar; si la aplicación se ralentiza Javascript esperará a que termine la repetición anterior para llamar de nuevo. También el cálculo del tiempo entre frames (explicado en el apartado \ref{fps_bucle_ejecucion}) se hará desde Javascript aunque esto no es estrictamente necesario.

Previamente a la llamada del bucle se llamará a una función ``Init", con el objetivo de incializar el motor y el contexto de WebGL

\subsection{Gestión de ficheros}


\cleardoublepage
\chapter{Diseño}
\label{app_design}
Antes de empezar el proceso de desarrollo se han tomado una serie de decisiones de diseño. Realizar este paso ayuda a tener una mejor idea de lo que se quiere hacer, cuales serán las necesidades y qué problemas pueden surgir. En este capítulo se describirá el diseño de la aplicación incluyendo los cambios que se han realizado durante el desarrollo.

%\section{Estructura del programa}
\section{Núcleo, incialización y gestión de inputs}
\label{world}
El elemento más importante de la aplicación, alrededor del cual girará el resto, es la clase ``World". Puede entenderse esta clase como la ``puerta de entrada" del flujo del motor hacia la aplicación. Es la primera clase de la aplicación en llamarse en la incialización, actualización y finalización del programa (véase el apartado \ref{fps_bucle_ejecucion} sobre el bucle de ejecución).

Desde ``World" se incializa la cámara, y las luces. Un fichero externo \texttt{.json} describe el estado inicial de la aplicación y a partir de esta configuración se inicializa el progrma. También es la primera clase de la aplicación en recibir la orden de actualización del motor y los inputs del usario y debe transmitirsela a los diferentes elementos que gestione.

También se encargará de actualizar en cada iteración los diferentes componentes de la aplicación, y gestionar los estados del sistema de estados (explicado en el apartado \ref{state_pattern}).

%%%%%%%%%%%%%%%%%%%%%%%%%%%%%%%%%%%%%%%%%%%%%%%%%%%%%%%%%%%%%%%%%%%%%%%%%%%%%
%%%%%%%%%%%%%%%%%%%%%%%%%%%%%%%%%%%%%%%%%%%%%%%%%%%%%%%%%%%%%%%%%%%%%%%%%%%%%
%%%%%%%%%%%%%%%%%%%%%%%%%%%%%%%%%%%%%%%%%%%%%%%%%%%%%%%%%%%%%%%%%%%%%%%%%%%%%

\section{Gestión y generación de entidades}
\label{managers}
Como se ha explicado en la sección \ref{engine_design}, la aplicación se comunica con el motor a través de la escena y las entidades. Un primer punto importante es mantener el control de dichas entidades para que no acaben esparcidas por el código.

Normalmente es suficiente con organizarlas en estructuras de datos dentro de la propia clase World (\ref{world}). Sin embargo, algunos casos específicos como las paredes y huecos (cuya generación se explica en el apartado \ref{walls_holes} o los interiores), son lo suficientemente complejos y numerosos como para tener su propio gestor.

Dentro cada gestor hay una o varias listas de elementos que se organizan según sus índices en la lista, y que se inicializan con un tamaño predefinido. Se ha decidido hacer de este modo para evitar que la lista cambie de tamaño constantemente, provocando reinicializaciones de memoria. Una consecuencia es que debe hacerse una estimación del número de elementos que habrá y controlar mediante aserciones que no se supere ese número. 

Como las paredes se definen en forma de conjuntos de paredes, para referirse a ellas se utilizan dos números enteros: la posición en la que empieza la primera pared y el número de paredes en el conjunto. Si cambia el tamaño de un conjunto de paredes al actualizarlo se reorganiza la lista de paredes para que no haya fragmentación, y se controla en todo momento el número de paredes en uso, puesto que el tamaño de las listas no varía nunca y leer datos no utilizados provocaría errores.

En el caso de elementos que sean atómicos, como las ventanas o los interiores, es suficiente con controlarlos con su índice.

Para evitar que desde fuera del gestor se manipulen los índices, y para tener un identificador más sencillo de manejar, los gestores crean sus propios identificadores en forma de número entero y los relacionan con los elementos de los arrays mediante mapas. Esto permite por ejemplo, que si la posición de una pared cambia (porque se ha borrado una pared anterior) su identificador externo, el que tiene el usuario, siga significando lo mismo. Si se utilizaran índices cambiar el orden de los elementos resultaría en índices incorrectos, de este modo se evita que externamente haya que preocuparse por ello. El sistema es muy similar al que se describe en el apartado \ref{manta_id_management} sobre la gestión de identificadores del motor Manta.

Aunque no es estrictamente necesario, los gestores de entidades están pensados para utilizarse a través del gestor de comandos \ref{command_pattern}.

%%%%%%%%%%%%%%%%%%%%%%%%%%%%%%%%%%%%%%%%%%%%%%%%%%%%%%%%%%%%%%%%%%%%%%%%%%%%%
%%%%%%%%%%%%%%%%%%%%%%%%%%%%%%%%%%%%%%%%%%%%%%%%%%%%%%%%%%%%%%%%%%%%%%%%%%%%%
%%%%%%%%%%%%%%%%%%%%%%%%%%%%%%%%%%%%%%%%%%%%%%%%%%%%%%%%%%%%%%%%%%%%%%%%%%%%%

\section{Sistema de estados}
\label{state_pattern}
La aplicación contará con diversos estados principales: vista estándar, vista en planta ortogonal y vista en planta lateral. Cada uno de estos estados tiene propiedades que le diferencian del resto, tienen cámaras distintas y reaccionan de un modo distinto a los inputs del usuario.

La solución más evidente a esto es utilizar un conjunto de condicionales para controlar lo que se hace en cada uno de los estados; pero cuando estos son largos y complejos, el resultado es un código demasiado complicado y difícil de mantener. Además de este modo es fácil cometer errores y que se realicen acciones del estado incorrecto. Para separar mejor cada estado se ha hecho uso del Patrón de Estado\footfullcite{game_programming_patterns}.

Con el Patrón de Estado se define una clase princpial llamada \texttt{Estado}, que define una serie de métodos abstractos: las acciones a realizar para un mismo evento según el estado. En el caso del planificador sus métodos principales son: \texttt{Update}, \texttt{In}, \texttt{Out}, \texttt{OnMouseMove}, \texttt{OnMouseClick}, \texttt{OnMouseDown}, \texttt{OnMouseUp} y \texttt{OnScroll}.

Para los inputs de teclado se ha creado un array de comandos (véase \ref{command_pattern} sobre el Patrón Comando), pudiendo asociar el input con un Comando. Desde la clase \texttt{Estado} se recibe el input de teclado y ejecuta el Comando correspondiente.

La clase \texttt{Estado} debe entonces extenderse mediante herencia, creando otra clase por cada estado que pueda tener la aplicación. Al cambiar entre estados se ejecutará el método \texttt{Out} del estado que cerramos y el método \texttt{In} del que entra. Estos estados son instanciados por ``World" y según el estado actual los inputs y el método \texttt{Update} se redirigen a la instancia correspondiente.

Para que el usuario final pueda cambiar de un estado a otro existirá un input controlado también por ``World".

%%%%%%%%%%%%%%%%%%%%%%%%%%%%%%%%%%%%%%%%%%%%%%%%%%%%%%%%%%%%%%%%%%%%%%%%%%%%%
%%%%%%%%%%%%%%%%%%%%%%%%%%%%%%%%%%%%%%%%%%%%%%%%%%%%%%%%%%%%%%%%%%%%%%%%%%%%%
%%%%%%%%%%%%%%%%%%%%%%%%%%%%%%%%%%%%%%%%%%%%%%%%%%%%%%%%%%%%%%%%%%%%%%%%%%%%%

\section{Gestor de comandos}
\label{command_pattern}
El Patrón de Comandos\footfullcite{game_programming_patterns} permite definir una serie de acciones atómicas. De modo similar a lo que se hace con el Patrón Estado (\ref{state_pattern}) se crea una clase padre llamada \texttt{Comando}. Desde esta clase pueden extenderse comandos que realicen acciones determinadas. Para ejecutar el comando debe crearse una instancia y llamar al método abstracto \texttt{Do} del comando, que se implementa en cada clase derivada.

Este patrón permite implementar la característica de ``deshacer" acciones con el método abstracto \texttt{Undo}. Del mismo modo que definimos una acción por comando podemos definir una contra-acción que deshaga lo hecho. Esto implica que cada vez que se ejecuta la acción debe guardarse el estado previo para poder recuperarlo, si queremos que el comando pueda deshacerse.

Cuando en un comando se ejecuta la acción, su instancia se añade a una lista de acciones realizadas, llamada \texttt{History}. \texttt{History} cuenta con un cursor (un número entero que indica una posición en la lista) que indica la acción en la que se encuentra actualmente, y al deshacer una acción el cursor desciende una posición. Si se desea rehacer una acción, se debe buscar la instancia siguiente al cursor y ejecutar de nuevo su acción.

En el caso de que se realice una acción diferente después de haber deshecho un comando, todos los comandos posteriores al cursor se borran, impidiendo que se puedan rehacer. Se trata sin embargo de un comportamiento habitual de la opción ``deshacer" en otras aplicaciones, por lo que esto no empeora la experiencia de usuario. Si se alcanza el límite de acciones, el historial de acciones empieza a borrar por las primeras posiciones.

Implementar un comando por cada acción posible en la aplicación es sensiblemente más complicado que realizar las acciones sin más, pero este patrón es una de las pocas formas efectivas de implementar la posibilidad de deshacer. Si se decide que es una característica importante para el programa, es importante empezar a utilizarlo desde el principio, puesto que tratar de incluir este patrón en un programa grande y complejo suele requerir una gran inversión de tiempo.

Los comandos se deben instanciar como un puntero, aunque desde fuera no es necesario gestionar su memoria. La implementación utilizada del Patrón Comando registra automáticamente todas las intancias (aunque no estén en el historial) y libera su memoria cuando es necesario.

Otra característica de la implementación utilizada es que puede ejecutarse un comando de forma ``silenciosa" sin que este se registre en el historial, mediante el método \texttt{Do\_Silent}. Entre otras cosas, esto puede utilizarse para deshacer acciones aprovechando otros comandos existentes: si por ejemplo tenemos la acción ``añadir" y ``borrar", el método \texttt{Undo} de cada uno de ellos puede hacerse con una llamada silenciosa al otro. Esto ahorra complejidad y, si un comando cambia, no es necesario buscar sus equivalentes por el código para reflejar los cambios.

La implementación de cada comando queda como responsabilidad del usuario, en el apartado \ref{use_of_command} se dan detalles del uso que se le ha dado al patrón.

%%%%%%%%%%%%%%%%%%%%%%%%%%%%%%%%%%%%%%%%%%%%%%%%%%%%%%%%%%%%%%%%%%%%%%%%%%%%%
%%%%%%%%%%%%%%%%%%%%%%%%%%%%%%%%%%%%%%%%%%%%%%%%%%%%%%%%%%%%%%%%%%%%%%%%%%%%%
%%%%%%%%%%%%%%%%%%%%%%%%%%%%%%%%%%%%%%%%%%%%%%%%%%%%%%%%%%%%%%%%%%%%%%%%%%%%%

\section{Clases contra espacios de nombre}
Una decisión importante de diseño ha sido la de no utilizar en ningún caso el patrón Singleton. El patrón Singleton limita a una la cantidad de instancias que se puede tener de la clase que lo aplica, lo cual impide que se haga un mal uso de la clase si su propósito era que fuese una clase de una única instancia. Para conseguirlo se hace que el constructor de la clase sea privado y que la clase tenga una referencia estática y privada de sí misma dentro; el Singleton se encarga de manejar su propia instancia.

Aunque es un patrón muy útil y sencillo, en la mayoría de casos puede resolverse el problema de un modo más simple, como utilizando una clase que solo tenga métodos estáticos o, en C++, con funciones dentro de espacios de nombre. Un espacio de nombre es un ámbito (más conocido como scope) que permite limitar el acceso a su contenido, no se puede llamar a ninguna de sus funciones ni acceder a sus variables si no se especifica antes su espacio de nombre. Desde fuera, puede utilizarse del mismo modo que se haría con una clase puramente estática, pero tiene algunas diferencias importantes:

\begin{itemize}
    \item Puede extenderse y definirse en ficheros distintos. Es algo que ayuda a la mantenibilidad si el código es extenso y complejo, puesto que es posible separar las diferentes partes. Por otro lado, una desventaja es que el programador usuario del espacio de nombre también podría definir lo que quisiera dentro de este, abriendo la puerta a ``hacks", aunque en estos casos se asume que la responsabilidad es de quien utiliza mal la herramienta.
    
    \item No tiene métodos ni atributos privados. Sin embargo, es posible emular el comportamiento de los elementos privados extendiendo la clase dentro de la especificación del código, ocultándolo por tanto de los ficheros de cabecera, que son los que permiten que un código sea accesible desde otros puntos del programa.
    
    \item No tiene herencia. No es posible utilizar herencia en espacios de nombre, es una característica única de las clases. Si la herencia es importante los espacios de nombre quedan descartados.
\end{itemize}

Otro de los beneficios de un espacio de nombre es que permite evitar colisiones de nombre entre funciones y variables. Por ejemplo, si una de las librerías utilizadas tuviera una función llamada ``max", sería posible redefinirla dentro de un espacio de nombre.

Al final se ha escogido esta opción como sustituto del patrón Singleton en todos los casos donde no se requiera herencia ni instancias.


\cleardoublepage
\chapter{Desarrollo}
\section{Geometría dinámica}
\label{walls_holes}
Uno de los principales requerimientos para un planificador es la visualización de paredes y configuración de estas para adaptarlas a las medidas de una estancia. Además, deben poderse ubicar ventanas y puertas en las paredes, lo cual afecta a la geometría de la pared dado que se debe poder ver a través de estos elementos. Para ello no es factible utilizar mallas pregeneradas, se necesita generar las paredes de forma dinámica.

En este apartado se describe el proceso de desarrollo de las paredes en 2 iteraciones, una primera en la que se crea una estructura básica para las paredes, y otra en la que se tienen en cuenta las ventanas y puertas.

%%%%%%%%%%%%%%%%%%%%%%%%%%%%%%%%%%%%%%%%%%%%%%%%%%%%%%%%%%%%%%%%%%%%%%%%%%%%%
%%%%%%%%%%%%%%%%%%%%%%%%%%%%%%%%%%%%%%%%%%%%%%%%%%%%%%%%%%%%%%%%%%%%%%%%%%%%%
%%%%%%%%%%%%%%%%%%%%%%%%%%%%%%%%%%%%%%%%%%%%%%%%%%%%%%%%%%%%%%%%%%%%%%%%%%%%%
\subsection{Generación de la estructura básica de la pared}
\label{subsec:gen1}
El primer paso ha sido crear una definición de los datos que se recibirán para describir cómo ha de ser la pared. Se trata de una lista de puntos en dos dimensiones y un valor booleano que indica si esta lista debe cerrarse conectando el último punto con el primero. Esto último es importante porque, como se puede ver en la figura \ref{fig:vertical_view_walls}, la geometría de una esquina ``suelta" es diferente a la de una esquina que conecta dos paredes entre sí.

\begin{figure}[h]
    \centering
    \includegraphics[width=0.65\linewidth]{Vista_Vertical_Paredes}
    \caption{Paredes en vista vertical.}
    \label{fig:vertical_view_walls}
\end{figure}

Estos datos nos permiten no sólo crear habitaciones sino también paredes únicas o incluso otros tipos de estructuras, normalmente interiores, similares a una pared. La intención es que en el futuro el programa pueda utilizarse en otras herramientas para hacer diseños más complicados como el plano de una planta completa de un edificio.

Teniendo en cuenta la estructura de una malla, explicada en el apartado \ref{mesh_light_cam}, en la figura \ref{fig:io_generatewalls} se puede ver la conversión de los datos que se espera conseguir.

\begin{figure}[H]
    \centering
    \includegraphics[width=0.5\linewidth]{IO_paredes}
    \caption{Input y output del generador de paredes.}
    \label{fig:io_generatewalls}
\end{figure}

El output está formado por listas de valores planos. Por ejemplo, la lista de vértices está formada por los valores de posición ``x,y,z" de cada uno sucesivamente.

Por cada pared hay tres primeros puntos 2D relevantes: las esquinas izquierda de las paredes anterior, actual, y siguiente. A partir de estos tres puntos puede deducirse la información de la figura \ref{fig:wall_vectors}, donde los vectores ``N1" y ``N2" son las normales de cada pared, siempre hacia el exterior de la estancia.

\begin{figure}[H]
    \centering
    \includegraphics[width=\linewidth]{data_from_3_points}
    \caption{Vectores extraídos a partir de 3 puntos consecutivos.}
    \label{fig:wall_vectors}
\end{figure}

``N1" y ``N2" pueden obtenerse normalizando los vectores de un punto a otro, y girándolos. La dirección del vector central es la suma normalizada de estos dos o, en el caso de las paredes en los extremos cuando el circuito no está cerrado, la normal de la propia pared:

\begin{lstlisting}
VARIABLES EXTERNAS:
    puntos: lista de puntos 2D
    i: punto desde el que queremos obtener las normales de pared
    cerrado: booleano que indica si se quiere cerrar el conjunto de paredes

VARIABLE direccion TIPO VECTOR
VARIABLES v_pc, v_cn, normal_actual, normal_anterior TIPO VECTOR
VARIABLES actual, anterior, siguiente TIPO NUMERO

actual = i
anterior = i - 1

SI i EQUIVALE A tamaño(puntos) ENTONCES:
    siguiente = 0
SINO
    siguiente = i + 1
FINSI

v_pc = puntos[actual] - puntos[anterior];
v_cn = puntos[siguiente] - puntos[actual];

SI cerrado Y (actual EQUIVALE A tamaño(puntos) - 1 O actual EQUIVALE A 0) ENTONCES:
    normal_actual = GIRAR 90 GRADOS ANTI-HORARIO v_cn Y NORMALIZAR
    direccion = puntos[actual] + normal_actual
SINO
    normal_actual = GIRAR 90 GRADOS ANTI-HORARIO v_cn
    normal_anterior = GIRAR 90 GRADOS ANTI-HORARIO v_pc
    direccion = normal_actual + normal_anterior
    NORMALIZAR direccion
FINSI
\end{lstlisting}

Por último se debe tener en cuenta el grosor que se espera que tenga la pared, dado que si se avanza siempre la misma distancia en el vector director el grosor de estas dependería del ángulo que formen con sus paredes adyacentes. Esto se resuelve con trigonometría:

\begin{lstlisting}
SI cerrado ENTONCES:
    direccion = profundidad_pared / absoluto(producto_punto(normal_actual, direccion));
SINO
    direccion = profundidad_pared * normal_actual
FINSI

punto_esquina = puntos[actual] + direccion
\end{lstlisting}

El producto punto de dos vectores normales es el coseno del ángulo que forman entre sí. En este momento ``direccion" incluye la dirección y distancia entre el punto actual y el punto de la esquina exterior, permitiendo generar dicho punto a partir del actual. En el caso de que la pared que estamos generando no haga esquina con otra, simplemente se utiliza la normal de la pared actual.

Una limitación de este sistema es que los puntos deben introducirse en sentido anti-horario respecto al interior de la habitación. De lo contrario la normal de cada pared queda invertida y estas se extienden hacia el lado opuesto al que deberían, provocando algunos artefactos no deseados.

%%%%%%%%%%%%%%%%%%%%%%%%%%%%%%%%%%%%%%%%%%%%%%%%%%%%%%%%%%%%%%%%%%%%%%%%%%%%%
%%%%%%%%%%%%%%%%%%%%%%%%%%%%%%%%%%%%%%%%%%%%%%%%%%%%%%%%%%%%%%%%%%%%%%%%%%%%%
%%%%%%%%%%%%%%%%%%%%%%%%%%%%%%%%%%%%%%%%%%%%%%%%%%%%%%%%%%%%%%%%%%%%%%%%%%%%%
\subsection{Índices para la estructura básica}
Para referir a cada uno de los vértices se ha definido la nomenclatura ``A, B, A2, B2" como puede verse en la figura \ref{fig:nomenclatura_vertices}, además de los correspondientes ``AH, BH, A2H, B2H" en la parte alta de la pared. Posteriormente, los índices de la pared se extraen de los que habría normalmente en un cubo, aprovechando que sus vértices se conectan del mismo modo.

\begin{figure}[H]
    \centering
    \includegraphics[width=0.75\linewidth]{Nomenclaturas_vertices}
    \caption{Nomenclatura básica de los vértices.}
    \label{fig:nomenclatura_vertices}
\end{figure}

La pared tiene 23 vértices y no 8 como cabría esperar. Esto se debe a que al renderizar, los shaders interpolan la normal de cada vértice con la de sus vecinos; si el mismo vértice se encuentra en dos caras distintas, la normal del vértice no coincide con la de la superficie en la cara que se está pintando. El resultado de esto sería que el color varía en los bordes de cada cara. Esta propiedad es muy útil para objetos que no tienen ángulos tan marcados, pero en este caso se busca que las caras sean muy marcadas y totalmente planas.

Para solucionarlo se repite cada vértice tantas veces como el número de caras en el que se encuentre, de modo que aunque todos se encuentren en la misma posición, cada cara está utilizando un vértice distinto.

\begin{figure}[H]
    \centering
    \includegraphics[width=0.75\linewidth]{paredes_first}
    \caption{Ejemplo de generación de paredes.}
    \label{fig:paredes_first}
\end{figure}

%%%%%%%%%%%%%%%%%%%%%%%%%%%%%%%%%%%%%%%%%%%%%%%%%%%%%%%%%%%%%%%%%%%%%%%%%%%%%
%%%%%%%%%%%%%%%%%%%%%%%%%%%%%%%%%%%%%%%%%%%%%%%%%%%%%%%%%%%%%%%%%%%%%%%%%%%%%
%%%%%%%%%%%%%%%%%%%%%%%%%%%%%%%%%%%%%%%%%%%%%%%%%%%%%%%%%%%%%%%%%%%%%%%%%%%%%
\subsection{Modificando la estructura para permitir la inclusión de ventanas}
\label{subsec:gen2}
El siguiente paso es implementar la posibilidad de añadir puertas y ventanas a la estancia. Insertar el modelo correspondiente en cada caso no será un problema, pero para que el efecto sea convincente es necesario poder ver a través de estos. Eso implica que se debe que modificar la geometría de las paredes para que incluya huecos donde tengan que ir dichas ventanas y puertas.

Del mismo modo que con las paredes inicialmente, se define un input: por cada hueco existe un punto 3D (que como se verá a continuación, no necesariamente debe colisionar con una pared), una altura y un ancho. El input/output del algoritmo completo para generar paredes quedaría como se puede ver en la figura \ref{fig:io_generatewindows}.

\begin{figure}[H]
    \centering
    \includegraphics[width=0.75\linewidth]{I_O_ventanas}
    \caption{Nuevo input y output de GenerateWalls, incluyendo ventanas.}
    \label{fig:io_generatewindows}
\end{figure}

Para empezar se deben adaptar las paredes que ya generadas. Es conveniente que la generación de ventanas se limite a trabajar sobre un solo plano, de modo que se eliminarán los planos anterior y posterior de la pared para añadirlos después con la nueva geometría. Aunque intuitivamente pueda parecer que esto supone simplificar la geometría respecto a lo que se ha hecho hasta ahora, en realidad se complica sensiblemente. Los planos anterior y posterior de la pared van a ser siempre idénticos, pero el plano posterior es algo más alargado debido a la geometría de las esquinas que se puede apreciar en la figura \ref{fig:nomenclatura_vertices}.

\begin{figure}[H]
    \centering
    \includegraphics[width=0.75\linewidth]{Nomenclaturas_vertices_2}
    \caption{Nomenclatura final de los vértices.}
    \label{fig:nomenclatura_vertices_2}
\end{figure}

Con la nueva geometría presentada en la figura \ref{fig:nomenclatura_vertices_2} las paredes se convierten en dos prismas triangulares unidos por dos planos superior e inferior de la pared. Con esto se consigue que los planos anterior y posterior que ahora le faltan a la pared sean totalmente idénticos aunque con las normales invertidas. Esto, sin embargo, complica las conexiones entre los vértices, que se han tenido que generar manualmente. Con los nuevos vértices añadidos en total hay 35 vértices.

\begin{figure}[H]
    \centering
    \includegraphics[width=0.75\linewidth]{paredes_frame}
    \caption{Aspecto de las paredes sin los planos anterior y posterior.}
    \label{fig:nueva_estructura}
\end{figure}

En la figura \ref{fig:nueva_estructura} puede verse el resultado. Nótese que los planos inferiores no se ven porque sus normales apuntan hacia abajo y el motor gráfico los oculta como optimización. Al crear los índices se ha tenido en cuenta el orden de estos, pues afecta al cálculo de las normales de los vértices.

%%%%%%%%%%%%%%%%%%%%%%%%%%%%%%%%%%%%%%%%%%%%%%%%%%%%%%%%%%%%%%%%%%%%%%%%%%%%%
%%%%%%%%%%%%%%%%%%%%%%%%%%%%%%%%%%%%%%%%%%%%%%%%%%%%%%%%%%%%%%%%%%%%%%%%%%%%%
%%%%%%%%%%%%%%%%%%%%%%%%%%%%%%%%%%%%%%%%%%%%%%%%%%%%%%%%%%%%%%%%%%%%%%%%%%%%%
%\clearpage
\subsection{Generación de ventanas I: proyección sobre pared}
\label{sec:wallgenwindowsi}
Como ya se ha mencionado en el apartado \ref{subsec:gen2}, las ventanas están definidas por un punto, una altura y un ancho. No se incluye ninguna información respecto a que pared es la que va a contener dicha ventana; por lo que se cogerá la pared más cercana al punto dado.

Para ello se proyecta el punto sobre la línea $AB$ de cada una de las paredes, calculando la distacia hasta cada una de las proyecciones para ver cuál es la más cercana (véase el apéndice \ref{sec:pointrayproj} sobre la proyección punto-línea):

\begin{lstlisting}
findWall(punto):
    VARIABLE pared_proxima
    VARIABLE distancia_menor
    VARIABLE distancia
    
    pared_proxima = -1
    distancia_menor = -1.0
    
    POR CADA ELEMENTO EN paredes, i:
        VARIABLE proyeccion
        proyeccion = proyeccion_punto_linea(pared[i].A1, pared[i].B1, punto)
        
        SI proyeccion encontrada ENTONCES:
            distancia = LOGITUD DE VECTOR (proyeccion - punto)
            SI distancia_menor ES MENOR QUE 0.0 O distancia < distancia_menor ENTONCES:
                distancia_menor = distancia
                pared_proxima = i
            FINSI
        FINSI
    FINPOR
    DEVOLVER pared_proxima
FIN DE FUNCION
\end{lstlisting}

Todas las ventanas o puertas han de ser necesariamente rectangulares. Esta es una precondición que se ha impuesto desde desarrollo para simplificar el cálculo de los agujeros, dado que es más sencillo incorporar geometrías más complejas incluyendo paredes falsas dentro del modelo de la ventana o puerta. Por ejemplo, si en algún momento se deseara incorporar una ventana redonda, sería más sencillo incluir 4 esquinas de pared a la ventana permitiendo que el hueco sea rectangular igualmente.

Como preparación para el próximo apartado (\ref{sec:wallgenwindowsii}) se calcula sobre la pared los 4 puntos que delimitarán la ventana. Para ello se proyecta el punto de la pared sobre el plano que forma esta, y se calcula el resto de puntos desplazándonos por dicho plano (véase el apéndice \ref{sec:pointplaneproj} sobre la proyección punto-rectángulo):

\begin{lstlisting}
projectVertices(pared, referencia a hueco):
    VARIABLE direccion_A1B1
    VARIABLE direccion_A1A1H
    VARIABLE origen
    
    direccion_A1B1 = NORMALIZAR (pared.B1 - pared.A1)
    direccion_A1A1H = NORMALIZAR (pared.A1H - pared.A1)
    
    origen = proyeccion_punto_rectangulo(pared.A1, pared.B1, pared.A1H, hueco.origen)
    
    hueco.A1 = origen
    hueco.B1 = origen + direccion_A1B1 * hueco.ancho
    hueco.A1H = origen + direccion_A1A1H * hueco.alto
    hueco.B1H = hueco.A1H + direccion_A1B1 * hueco.ancho
FIN DE FUNCION
\end{lstlisting}

%%%%%%%%%%%%%%%%%%%%%%%%%%%%%%%%%%%%%%%%%%%%%%%%%%%%%%%%%%%%%%%%%%%%%%%%%%%%%
%%%%%%%%%%%%%%%%%%%%%%%%%%%%%%%%%%%%%%%%%%%%%%%%%%%%%%%%%%%%%%%%%%%%%%%%%%%%%
%%%%%%%%%%%%%%%%%%%%%%%%%%%%%%%%%%%%%%%%%%%%%%%%%%%%%%%%%%%%%%%%%%%%%%%%%%%%%
\subsection{Generación de ventanas II: modificación de la geometría de la pared}
\label{sec:wallgenwindowsii}

Al final del apartado \ref{sec:wallgenwindowsii}, ya se conoce el punto en que están los extremos de la ventana sobre la pared. En este apartado se explica cómo dividir la pared en diferentes planos y descartar aquellos que correspondan a un agujero.

Una vez más las proyecciones cobran mucho protagonismo (véase el apéndice \ref{sec:pointrayproj} sobre la proyección punto-línea). Lo primero que se busca es dividir el plano de la pared por los vértices del hueco, obteniendo una colección de planos como se puede ver en la figura \ref{fig:wall_separacion}.

\begin{figure}[H]
    \centering
    \includegraphics[width=0.85\linewidth]{Separaciones_paredes}
    \caption{Separación de la pared en planos}
    \label{fig:wall_separacion}
\end{figure}

El algoritmo para conseguir esto tiene como entrada y salida una lista de planos, que empieza siendo uno solo que cubre toda la pared. Mientras iteramos los puntos, proyectamos estos sobre cada lado para obtener los puntos por los que hay que cortar los planos, y posteriormente se añaden a la lista de planos desechando el original. En caso de que una de las proyecciones no contribuya a crear un nuevo plano (como por ejemplo, los puntos inferiores de la puerta en la figura \ref{fig:mult_and_red_windows}) la ignoramos.

Posteriormente se comprueba cuales de estos planos forman parte de una ventana y se eliminan de la lista, dejando un hueco en dicha posición. Este algoritmo permite además añadir múltiples ventanas, aumentando la posible complejidad de la pared.

Por último se reprocesan los planos generados comprobando si sus lados coinciden en alguna dirección, en cuyo caso se reúnen para reducir la complejidad. En la figura \ref{fig:mult_and_red_windows}, se puede ver un ejemplo de como se separan los planos con múltiples ventanas, y cómo se reúnen los planos adyacentes.

\begin{figure}[H]
    \centering
    \includegraphics[width=0.85\linewidth]{mult_window_and_reduction}
    \caption{Ejemplo de pared con una ventana y una puerta, y muestra de una posible reducción de los planos.}
    \label{fig:mult_and_red_windows}
\end{figure}

\begin{figure}[H]
    \centering
    \includegraphics[width=0.85\linewidth]{Agujeros}
    \caption{Ejemplo de generación de paredes con huecos y estancia sin cerrar.}
    \label{fig:wall_with_window_example}
\end{figure}

%%%%%%%%%%%%%%%%%%%%%%%%%%%%%%%%%%%%%%%%%%%%%%%%%%%%%%%%%%%%%%%%%%%%%%%%%%%%%
%%%%%%%%%%%%%%%%%%%%%%%%%%%%%%%%%%%%%%%%%%%%%%%%%%%%%%%%%%%%%%%%%%%%%%%%%%%%%
%%%%%%%%%%%%%%%%%%%%%%%%%%%%%%%%%%%%%%%%%%%%%%%%%%%%%%%%%%%%%%%%%%%%%%%%%%%%%
\subsection{Generación de uvs}
Para que el motor aplique correctamente las texturas sobre las paredes, se requiere que las uvs estén a escala de mundo; es decir, se espera que dada una distancia entre dos vértices de la mima malla, sus uvs tengan la misma distancia. La mayor dificultad en este caso está en que los vértices se encuentran en espacio tridimensional, mientras que las uv son coordenadas bidimensionales de la textura que estamos mapeando. Por lo tanto, de algún modo hay que desconsiderar la orientación de la pared y centrarnos en sus superficies.

Como la geometría está formada por planos perfectos, la solución encontrada ha consistido en partir siempre de la esquina inferior izquierda de cada uno de ellos. Como precondición, la esquina ``A" tiene la uv (0,0).

Por lo tanto por cada plano, se ha definido la uv de la esquina inferior izquierda, y los vectores hacia los cuales ``avanzan" las componentes \texttt{x} e \texttt{y} de las uv (Fig. \ref{fig:datos_uvs}).

\begin{figure}[H]
    \centering
    \includegraphics[scale=0.75]{Datos_uvs}
    \caption{Datos necesarios para la generación de las uv ``2" y ``3".}
    \label{fig:datos_uvs}
\end{figure}

La razón de usar proyecciones y no la magnitud de los vectores directamente, es que se desconoce en cual de las dos direcciones se encuentra el vértice que observamos al iterar. Proyectando cada punto tenemos la distancia en cada una de las dos direcciones y se la sumamos a la uv del vértice ``1":

\begin{lstlisting}
genUV(origen, xDir, yDir, punto, uv_origen, longitud_x, longitud_y):
    VARIABLE uv COMO VECTOR 2D
    uv.x = proyeccion_punto_rayo(origen, xDir, punto)
    uv.y = proyeccion_punto_rayo(origen, yDir, punto)
    
    SI longitud_x > 0.0 Y uv.x < 0.0 ENTONCES:
        uv.x = longitud_x - uv.x
    FINSI
    SI longitud_y > 0.0 Y uv.y < 0.0 ENTONCES
        uv.y = longitud_y - uv.y
    FINSI
    
    uv = uv + uv_origen
    DEVOLVER uv
FIN DE FUNCION
\end{lstlisting}

Dado que las uv no pueden ser negativas, se hace un paso en las líneas 6-11 para que, en caso de serlo, se les sume la longitud total del lado en el que se encuentran.

%%%%%%%%%%%%%%%%%%%%%%%%%%%%%%%%%%%%%%%%%%%%%%%%%%%%%%%%%%%%%%%%%%%%%%%%%%%%%
%%%%%%%%%%%%%%%%%%%%%%%%%%%%%%%%%%%%%%%%%%%%%%%%%%%%%%%%%%%%%%%%%%%%%%%%%%%%%
%%%%%%%%%%%%%%%%%%%%%%%%%%%%%%%%%%%%%%%%%%%%%%%%%%%%%%%%%%%%%%%%%%%%%%%%%%%%%
\subsection{Generación de normales, tangentes y bitangentes}
El cálculo de cada normal es el resultado de la suma de la normal de cada polígono en el que se encuentra dicho vértice, y para calcular esta se hace el producto vectorial (normalizado) de los vectores que van del primer vértice hacia el segundo y el tercero:

\begin{lstlisting}
VARIABLES EXTERNAS v1, v2, v3

normal = NORMALIZAR ( PRODUCTO CRUZADO DE (v2 - v1) Y (v3 - v1))
\end{lstlisting}

Las tangentes y bitangentes son un poco más complicadas: todo vector tiene infinitos vectores tangentes, pero en este caso no sirve cualquiera. El vector tangente a la normal tiene que estar siempre alineado con las uv.

Como se explica en el tutorial 13 de opengl-tutorial.org \footfullcite{opengltutorials}, para ello debemos resolver el siguiente sistema de ecuaciones:


\[ deltaPos1 = deltaUV1.x * T + deltaUV1.y * B \]
\[ deltaPos2 = deltaUV2.x * T + deltaUV2.y * B \]

Esto se computar como

\begin{lstlisting}
VARIABLES EXTERNAS uv1, uv2, uv3, v1, v2, v3

VARIABLES vec1, vec2
vec1 = v2 - v1
vec2 = v3 - v1

VARIABLES deltaUV1, deltaUV2 COMO VECTORES 2D
deltaUV1 = uv2 - uv1
deltaUV2 = uv3 - uv1

VARIABLE r
r = 1.0 / (deltaUV1.x * deltaUV2.y - deltaUV1.y * deltaUV2.x)

VARIABLES tangente, bitangente
tangente = (vec1 * deltaUV2 - vec2 * deltaUV1.y) * r
bitangente = (vec2 * deltaUV1.x - vec1 * deltaUV2.x) * r
\end{lstlisting}


\cleardoublepage
\chapter{Conclusiones}

\cleardoublepage
\appendix
\chapter{Utilidades matemáticas}
Normalmente un motor gráfico incluye una serie de herramientas para facilitar el desarrollo, pero debido a las carencias de nuestro motor en este aspecto, he tenido que programarlas yo mismo. Este apéndice se presenta como referencia para los fragmentos de código que hacen uso de estas herramientas a lo largo de este documento, y como muestra del funcionamiento de estas.

%%%%%%%%%%%%%%%%%%%%%%%%%%%%%%%%%%%%%%%%%%%%%%%%%%%%%%%%%%%%%%%
%%%%%%%%%%%%%%%%%%%%%%%%%%%%%%%%%%%%%%%%%%%%%%%%%%%%%%%%%%%%%%%
%%%%%%%%%%%%%%%%%%%%%%%%%%%%%%%%%%%%%%%%%%%%%%%%%%%%%%%%%%%%%%%
\section{Comparación de tipos imprecisos}
En la mayoría de dispositivos los números de coma flotante tienen un cierto nivel de imprecisión. Esto no suele ser un problema porque suelen tener mucha más precisión de la necesaria, y en su defecto hay otras formas de conseguir aún más precisión. El problema es no podemos comparar dichos números porque rara vez van a ser \textit{exactamente} idénticos. Para ello he creado las funciones:

\begin{lstlisting}
bool compare_float(float A, float B, float precission = 0.01f);
bool compare_vec(glm::vec3, glm::vec3, float precission = 0.01f);
\end{lstlisting}

La cual sencillamente compara que el valor absoluto de la diferencia entre los dos números sea inferor a la precisión requerida (por defecto $0.01$). Como los vectores en glm utilizan números de coma flotante, tenemos que hacer lo mismo para poder compararlos, pero esta vez comparando sus 3 componentes.

%%%%%%%%%%%%%%%%%%%%%%%%%%%%%%%%%%%%%%%%%%%%%%%%%%%%%%%%%%%%%%%
%%%%%%%%%%%%%%%%%%%%%%%%%%%%%%%%%%%%%%%%%%%%%%%%%%%%%%%%%%%%%%%
%%%%%%%%%%%%%%%%%%%%%%%%%%%%%%%%%%%%%%%%%%%%%%%%%%%%%%%%%%%%%%%
\section{Proyección punto-rayo y punto-línea}
\label{sec:pointrayproj}
Entendiendo un rayo como un elemento formado por un punto y una dirección y una línea como un segmento de un rayo delimitado por dos puntos, he creado las siguientes dos funciones:

\begin{lstlisting}
float point_ray_projection(glm::vec3 ray_origin, glm::vec3 ray_direction, glm::vec3 point);
bool point_line_projection(glm::vec3 line_A, glm::vec3 line_B, glm::vec3 point, glm::vec3& result);
\end{lstlisting}

Su funcionamiento es muy similar, de hecho la segunda hace uso de la primera para obtener el resultado, pero tienen dos diferencias importantes: ``point\_ray\_projection" devuelve la distancia entre el origen del rayo y la proyección de nuestro punto, en la dirección especificada, mientras que ``point\_line\_projection" devuelve un booleano que indica si la proyección está dentro de nuestra línea, y asigna a la referencia ``result" el punto exacto de la proyección.

Hay diferentes situaciones en las que puede ser más útil una u otra: a veces querremos saber el punto exacto de la proyección, otras veces querremos saber la distancia de esta proyección respecto al punto de origen (sin necesidad de calcular el punto), y otras comprobar si esta proyección se encuentra delimitada entre dos puntos.

El cálculo de la proyección rayo-punto se basa en las propiedades del producto punto. A continuación explicaré el razonamiento por el cual sé que el producto punto puede usarse para calcular una proyección entre dos vectores:

El producto punto, o producto escalar de dos vectores, cumple la siguiente fórmula:
\[ dot(A,B) = |A||B|*cos(\Theta) \]

Donde $\theta$ es el ángulo que forman los dos vectores. Si asumimos que A es un vector normal (y nos aseguraremos de que así sea siempre), esto se reduce a:
\[ dot(|A|,B) = |B|*cos(\Theta) \]

Si miramos esta ecuación desde el punto de vista trigonométrico, $B$ puede entenderse como la hipotenusa del triángulo que forman $A$, $B$, y el vector de $B$ a la proyección de $B$ en $A$. El coseno, por definición, nos indica el ratio entre el lado contiguo de una esquina y la hipotenusa de un triángulo, por lo que al multiplicarlo por la magnitud de B obtenemos la longitud del lado contiguo, es decir, la distancia entre $A$ y la proyección de $B$.

Una vez calculada esta distancia, podemos sacar el punto de proyección fácilmente:
\[ P = origen + |direccion| * distancia \]

Todo esto se ha implementado de la siguiente manera:

\begin{lstlisting}
float Utils::point_ray_projection(
    glm::vec3 ray_origin, glm::vec3 ray_direction, glm::vec3 point
) {
	ray_direction = glm::normalize(ray_direction);
	return glm::dot(ray_direction, point - ray_origin);
}

bool Utils::point_line_projection(
    glm::vec3 A, glm::vec3 B, glm::vec3 point, glm::vec3 & result
) {
	glm::vec3 direction = glm::normalize(B - A);
	float projection = point_ray_projection(A, direction, point);
	result = A + direction * projection;
	float length = glm::length(B - A);
	return (
		(compare_float(projection, 0.0f) || projection > 0) && 
		(compare_float(projection, length) || projection < length)
	);
}
\end{lstlisting}

%%%%%%%%%%%%%%%%%%%%%%%%%%%%%%%%%%%%%%%%%%%%%%%%%%%%%%%%%%%%%%%
%%%%%%%%%%%%%%%%%%%%%%%%%%%%%%%%%%%%%%%%%%%%%%%%%%%%%%%%%%%%%%%
%%%%%%%%%%%%%%%%%%%%%%%%%%%%%%%%%%%%%%%%%%%%%%%%%%%%%%%%%%%%%%%
\clearpage
\section{Proyección punto-plano y punto-rectángulo}
\label{sec:pointplaneproj}
De un modo similar a lo que hemos visto en el apartado \ref{sec:pointrayproj}, podemos calcular la proyección a partir del producto punto.

Para ello debemos disponer de la normal de dicho plano. En el caso de la proyección punto-plano la requerimos como parámetro para definir el plano, pero en el caso de la proyección punto-rectángulo la calcularemos a partir de 3 puntos del plano. Este detalle es importante porque el orden de dichos puntos afectará a la dirección de la normal; no podemos hacer nada al respecto más que alertar a quien utilice esta función.

Dados 3 puntos $A$, $B$ y $C$, la normal del plano que forman es el producto vectorial de los vectores que van de un punto hacia los otros dos, normalizados, por ejemplo $N = AB \times AC$.

Teniendo esto en cuenta:
\begin{lstlisting}
glm::vec3 Utils::point_plane_projection(
    glm::vec3 origin, glm::vec3 normal, glm::vec3 point
) {
	normal = glm::normalize(normal);
	float dist = glm::dot(point - origin, normal);
	return point - normal * dist;
}

glm::vec3 Utils::point_rec_projection(
    glm::vec3 A, glm::vec3 B, glm::vec3 C, glm::vec3 point
) {
	glm::vec3 normal = glm::normalize(glm::cross(B - A, C - A));
	return  point_plane_projection(A, normal, point);
}
\end{lstlisting}


%%%%%%%%%%%%%%%%%%%%%%%%%%%%%%%%%%%%%%%%%%%%%%%%%%%%%%%%%%%%%%%
%%%%%%%%%%%%%%%%%%%%%%%%%%%%%%%%%%%%%%%%%%%%%%%%%%%%%%%%%%%%%%%
%%%%%%%%%%%%%%%%%%%%%%%%%%%%%%%%%%%%%%%%%%%%%%%%%%%%%%%%%%%%%%%
\section{Comprobar si cuatro puntos forman parte del mismo plano}
\label{sec:check4pointplane}
En un espacio de $D$ dimensiones, $D+1$ puntos forman parte del mismo espacio de $D-1$ dimensiones si el determinante de la matriz formada por las posiciones de los puntos organizadas verticalmente es 0 \footfullcite{points_sameplane}. En 3D:

\[
\begin{vmatrix}
& x_1 & x_2 & x_3 & x_4 &\\ 
& y_1 & y_2 & y_3 & y_4 &\\ 
& z_1 & z_2 & z_3 & z_4 &\\ 
& 1   & 1   & 1   & 1   &
\end{vmatrix} = 0
\]

Como glm ya dispone de una implementación para calcular el determinante de una matriz, ha bastado con generar la matriz a partir de los vectores y comprobar si su determinante es 0.

%%%%%%%%%%%%%%%%%%%%%%%%%%%%%%%%%%%%%%%%%%%%%%%%%%%%%%%%%%%%%%%
%%%%%%%%%%%%%%%%%%%%%%%%%%%%%%%%%%%%%%%%%%%%%%%%%%%%%%%%%%%%%%%
%%%%%%%%%%%%%%%%%%%%%%%%%%%%%%%%%%%%%%%%%%%%%%%%%%%%%%%%%%%%%%%
\section{Comprobar si la proyección de un punto sobre un plano está dentro de un rectángulo}
Aunque es similar, este se diferencia del apartado \ref{sec:check4pointplane} en que no se requiere que el punto que queremos comparar esté en el mismo plano que el resto.

Para ello utilizamos la proyección punto-línea explicada en el apartado \ref{sec:pointrayproj}. Teniendo un rectángulo definido por 3 esquinas $A$, $B$, $C$, un punto $P$ está dentro del rectángulo si podemos proyectarlo satisfactoriamente sobre las líneas $AB$ y $AC$:

\begin{lstlisting}
bool Utils::in_rec(
    glm::vec3 A, glm::vec3 B, glm::vec3 C, glm::vec3 point
) {
	glm::vec3 proj1, proj2;
	return 
		point_line_projection(A, B, point, proj1) && 
		point_line_projection(A, C, point, proj2);
}
\end{lstlisting}


\chapter{Patrones de diseño}
\section{Patrón componente}
\section{Patrón comando}
\section{Patrón estado}

%\bibliographystyle{unsrt}
%\bibliography{Bibliography}
\printbibliography

\end{document}